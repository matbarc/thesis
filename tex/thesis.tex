\documentclass[a4paper]{report}

% packages
\usepackage{amsmath, amssymb, mathtools, actuarialsymbol}

% bibliography
\usepackage[backend=biber, style=apa]{biblatex}
\addbibresource{thesis.bib}
\DeclareLanguageMapping{english}{english-apa}

% commands
\newtheorem{definition}{Definition}
\newcommand{\q}[2]{\prescript{}{#1}{q}_{#2}}
\newcommand{\p}[2]{\prescript{}{#1}{p}_{#2}}
\newcommand{\disc}[2]{\prescript{}{#1}{\mathcal{V}}_{#2}}

\title{Untitled: A Life Insurance forecasting tool} 
\author{Matheus Barcellos}

\begin{document}
  \maketitle  

  \begin{abstract}
    This will be an abstract.
  \end{abstract}

  \tableofcontents

  \chapter*{Introduction}
  
  In response to the ever-increasing computing power and its
   accessibility, unsurprisingly, software is increasingly 
  being deployed as tentative solution to simpify our lives,
   be it business or personal, to mixed results.
  Especially in the business side of things, it feels like 
  the complexity has simply been moved to the software 
  layer.
  The user is no longer required to understand the 
  intricacies of the real-world process being modelled, 
  but in turn is asked to be acquainted with the 
  ever-changing intricacies of whatever software is being 
  used, which might prove to be a questionable trade-off.
  
  The actuarial world is no different, 
  with companies trying to reap as many productivity gains 
  from the transition to the ever-more-fashionable choice 
  of third-party vendored software.
  These usually present an issue to the company making the 
  transition: to accomodate as many users as possible 
  the developers tend to pack them with features to the 
  point of introducing needless complexity, forcing the 
  company to invest serious capital (both time and money) 
  to achieve the advertised productivity gains.
  Large operations can easily absorb such costs and even 
  marginal productivity increases here and there can really 
  add up in this scale. Moreover, large operations tend to 
  pick up several quirks in their deals simply by virtue of 
  the volume of business, which warrants the complexity in 
  the software layer.
 
  TODO: Last Paragraph

  \chapter{Theoretical Framework}

    \section{Survival Models}
    \subsection{Mortality} 
    \label{mortality}
    (Pretty much this whole section is out of \textcite{actuarial}) 
    \\
    Let (x) denote a life aged x for $x \ge 0$.
    
    \begin{definition}
      Let the random variable $T_x$ denote the future 
      lifetime of (x) such that $x + T_x$ is equal to 
      its full lifespan.
      Define, also, the distribution function $F_x(t)$ 
      which represents the probability that (x) will not 
      survive more than t years, namely:
      
      \begin{equation}
	 F_x(t) = Pr(T_x \le t)
      \end{equation}
    \end{definition}
      
    In simpler terms, $F_x(t)$ is equal to the probability 
    that $(x)$ dies between ages x and t.
    Though this construct is incredibly important, 
    it is often more useful for actuaries to think of it 
    in terms of its counterpart, 
    the \textbf{survival function}.

    \begin{definition}[Survival Function]
      Define the survival function $S_x(t)$ to denote the 
      probability that (x) survives t years:
      
      \begin{equation}
	S_x(t) = Pr(T_x \ge t) = 1 - F_x(t)
      \end{equation}
    \end{definition}

   Finally, to complete our model we make the following
   assumption 
   (standard, more discussion on its validity in 
   Dickson, Hardy and Waters (2020)):
   \begin{equation}
     Pr(T_x \ge t) = Pr(T_0 \ge x + t \mid T_0 > x)
   \end{equation}
   Now equipped with this assumption and the following fact
   from basic probability:
   \[
     Pr(A \mid B) = \frac{Pr(A \cap B)}{Pr(B)}
   .\] 
   we can assert a very convenient relationship:
   \[
     Pr(T_x \ge t) = \frac{Pr(T_0 \ge x + t)}{Pr(T_0 > x)}
   .\] 
   Or, more conveniently expressed as:
   \begin{equation}
     S_x(t) = \frac{S_0(x+t)}{S_0(x)}
     \quad \text{or} \quad
     S_0(x+t) = S_0(x)S_x(t)
   \end{equation}
   This result is very important but it can be even further 
   extended.
   Consider:
   \begin{equation}
     S_x(t+k) = \frac{S_0(x+t+k)}{S_0(x)}
     = \frac{S_0(x+t)}{S_0(x)}\frac{S_0(x+t+k)}{S_0(x+t)} \\
     = S_x(t)S_{x+t}(k)
   \end{equation}
   In sum, equation (1.4) tells us that the probability 
   of survival from birth to an age $x+t$ can be split up
   as the product of the probability of surviving from 
   birth to $x$ and the probability of surviving from $x$
   to $x+t$.
   Equation (1.5) expands on this result by lifting the 
   requirement of the starting age being 0.
   We can, then, represent the probability of survival from 
   age $x$ to $x+k$ as the product of the probabilities of
   survival from $x$ to $x+t$ and from $x+t$ to $x+k$ for 
   some $t < k$.

   \subsection{International Actuarial Notation}
   \label{notation}
   The notation here onward used, unless otherwise noted,
   is in accordance with the so-called 
   \textit{International Actuarial Notation}. 
   This set of conventions allows for a greatly increased 
   expressivity in formulas and expressions while mostly 
   maintaining their succintness, in exchange for a slight 
   increase in complexity and recognizability.
   Appropriate notation will be introduced gradually in 
   tandem with the associated concepts, but let us define 
   the following as a foundation:
   \begin{align}
     \p{t}{x} &= S_x(t) \\
     \q{t}{x} &= 1 - S_x(t) = F_x(t) \\
     \q{k \mid t}{x} &= \p{k}{x}\q{t}{x+k} 
     = S_x(k) - S_x(k+t)
   \end{align}
   By convention, also, the right subscript may be dropped 
   in the case that it is 1. For instance:
   \[
     \q{1}{x} = \q{}{x} 
     \quad \text{and} \quad 
     \p{1}{x} = \p{}{x}
   .\] 

   \subsection{Curtate Lifetime}
  
   So far our modelling of mortality, based on the random 
   variable defined on section \ref{mortality}, 
   has been continuous. 
   However, to better represent the reality of premium/claim 
    payments this random variable needs to be adapted to 
   a discrete paradigm.
   Let this new random variable $K_x$ represent the integer 
   part of $T_x$, like:
    \[
   K_x = \left\lfloor T_x \right\rfloor,
   \]
   where "$\lfloor x \rfloor$" denotes the floor function.
   From the above relation we can, then, assert the 
   following:
   \begin{equation}
     Pr(K_x = k) = Pr(k \le T_x < k+1).
   \end{equation}
   Which we can rewrite it using our newly developed 
   notation: 
   \begin{align*}
     Pr(K_x=k) &= Pr(k \le T_x) - Pr(T_x > k+1) \\
               &= \p{k}{x} - \p{k+1}{x} \\
               &= \p{k}{x}\q{}{x+k}
   \end{align*}
    , which culminates in the following definition.

    \begin{definition}[Curtate Lifespan]
      Let the Curtate Lifespan
      $K_x = \left\lfloor T_x \right\rfloor$  be a discrete 
      random variable representing the lifespan of a life
      (x) in integer years (rounded down), 
      defined by the following distribution: 
      \begin{equation}
	Pr(K_x = k) = \p{k}{x}\q{}{x+k} = \q{k \mid}{x}
      \end{equation}.
    \end{definition}

   \subsection{Life Tables}
   
   TODO.

  \section{Discounting}
  The discounting model used here onwards is a pretty 
  standard discrete period discounting.
  As a refresher: we can find the value of a cashflow of C
  t periods in the future by the following relation: 
  \begin{equation}\label{fv}
    FV = C(1+i_0)(1+i_1)\ldots(1+i_{t-1})
  \end{equation}
  where $i_t$ is the interest rate for period $t$.
  We call this the \textit{future value} of the cashflow.
  Usually, in this sort of model interest is assumed to be 
  constant and yields the perhaps more recognizable formula:
   \[
     FV = C(1+i)^{t}
  ,\]
  where $i$ is this fixed rate.
  For the sake of customization, however, this assumption 
  will not be made prematurely, so Equation \ref{fv} will 
  be used.
  Similarly, we can have a notion of \textit{present value}
  as follows:
  \begin{equation}
    PV = \frac{C}{(1+i_0)\ldots(1+i_{t-1})} 
    = C(v_0)\ldots(v_{t-1})
  \end{equation}
  , where the $v_t=(1+i_t)^{-1}$.
  Up until now, notation has been fairly standard, but for 
  the sake of brevity in the coming chapters let us make
  the following, admittedly non-standard, definition:
  \begin{definition}
    Let $v_t$ stand for the discounting factor for period 
    $t$. Now, define $\disc{t}{x}$ to be the discounting 
    factor from time $x+t$ back to time $x$, such that:
    \begin{equation}
      \disc{t}{x} = (v_x)\ldots (v_{t-1}) 
      = \prod_{i=x}^{t-1} v_i
    \end{equation}
  \end{definition}
  
  For a more thorough discussion of the Theory of Interest 
  one can consult \textcite{kellison}.

  \section{Insurance}

   TODO. 

  \section{Annuities}
  
  These products function much like their certain 
  counterparts except, of course, for being contingent on 
  the benefitiary being alive at the time of the payment, 
  requiring us to do mortality discounting on top of 
  interest discounting. 
  Moreover, just like insurance, there is no way to 
  calculate a present value since they are contingent on a 
  random variable, so we can only deal with an expected 
  present value.

  \subsection{Whole Life}
  \begin{definition}
    Let $a_x$ denote the expected present value of a 
    whole-life annuity immediate to a life (x), 
    meaning every year (x) survives they will get a payment,      starting a year from inception. Then:
    \[
      a_x = a_{\angl{K_x}} 
      = \sum^{\infty}_{k=x} \left(\disc{k}{x}\right)\p{k}{x}
    .\] 
  \end{definition}

  \subsection{Term}
  \begin{definition}
    Let $a_{x:\angl{n}}$ denote the expected present value 
    of a term life annuity immediate to a life (x), 
    meaning every year (x) survives up to and including $x+n$
     they will get a payment, starting a year from inception.	 Then:
    \begin{align*}
      a_{x:\angl{n}} = 
      \begin{cases}
	a_{\angl{K_x}}, \phantom{n-} K_x < n \\
	a_{\angl{n}}, \phantom{K_x-} K_x \ge n
      \end{cases}
      = \sum^{x+n}_{k=x} \left(\disc{k}{x}\right)\p{k}{x}
    \end{align*}
  \end{definition}
  
  \section{Premiums}

    TODO.

  \section{Reserves}

  TODO (If there is enough time.)

  \chapter{Methodology and Algorithms}

  \section{Insurance}

    TODO.

  \section{Annuities}

    TODO.

  \section{Premiums}

    TODO.

  \section{Reserves}

  TODO (Contingent on 1.6)

  \section*{Conclusion}
  \printbibliography
\end{document}
